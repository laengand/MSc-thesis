\documentclass[onecolumn]{article}
\usepackage[top=3cm, bottom=3cm, left=3cm, right=2cm]{geometry}

\usepackage[english]{babel}
\usepackage[utf8]{inputenc}
\usepackage[T1]{fontenc}
\usepackage{amsmath}
\usepackage{verbatim}
\usepackage{color}
\usepackage[font=small]{caption}
\usepackage{subcaption}
%\usepackage{setspace}					% martin kan ikke bruge den
\usepackage{multicol}
\usepackage[T1]{fontenc}

\newcommand{\hl}[1]{\colorbox{yellow}{#1}}					% Gul baggrund

\linespread{1.5}						% linjeafstand
\usepackage{float}

\usepackage[pdftex]{graphicx}
\usepackage{wrapfig}
\usepackage[final]{pdfpages}
\graphicspath{	
	{./images/}	
}	
\renewcommand{\thefigure}{\thesection.\arabic{figure}}		% Figurer får nummering 4.1

\usepackage{tabularx}
\newcolumntype{L}{>{\raggedright\arraybackslash}X}
\newcolumntype{C}{>{\centering\arraybackslash}X}
\newcolumntype{R}{>{\raggedleft\arraybackslash}X}

\usepackage{tocloft}
\setlength\cftparskip{-2pt}
\setlength\cftbeforesecskip{1pt}
\setlength\cftaftertoctitleskip{2pt}

% \renewcommand{\arraystretch}{1.5}								% Rækkehøjden i tabeller
\newenvironment{Table}{\begin{table}[htbp] \footnotesize \centering}{\end{table}}

\usepackage{appendix}
\renewcommand{\contentsname}{Content}
\renewcommand{\appendixtocname}{Appendix}
\newcommand{\degree}{\ensuremath{^\circ}}

\setcounter{secnumdepth}{3}									% 1.1.1
\setcounter{tocdepth}{3}									% Viser ned til subsection i TOC


\setlength{\parskip}{0pt}									% Afstand imellem afsnit
\setlength{\parindent}{0pt}									% Første linje skal ikke rykke ind
\usepackage{hyperref}
\hypersetup{%
    pdfborder = {0 0 0}
}


\newcommand{\toc}{\pagestyle{empty} \setlength{\parskip}{0pt} \tableofcontents \setlength{\parskip}{10pt} \newpage \pagestyle{plain}}
\newcommand{\Section}[1]{\section{#1} \setcounter{figure}{0}}
\newcommand{\Subsection}[1]{\subsection{#1}}
\newcommand{\Subsubsection}[1]{\subsubsection{#1}}

\usepackage{booktabs}
\usepackage{listings}
\usepackage{wrapfig}

\usepackage{lastpage}
\usepackage{listings}             % Include the listings-package
\usepackage{color}

\usepackage{todonotes}		% Include todo (\todo{input}

\definecolor{mygreen}{rgb}{0,0.6,0}
\definecolor{mygray}{rgb}{0.5,0.5,0.5}
\definecolor{mymauve}{rgb}{0.58,0,0.82}

\lstset{ %
  backgroundcolor=\color{white},   % choose the background color; you must add \usepackage{color} or \usepackage{xcolor}
  basicstyle=\footnotesize,        % the size of the fonts that are used for the code
  breakatwhitespace=false,         % sets if automatic breaks should only happen at whitespace
  breaklines=true,                 % sets automatic line breaking
  captionpos=b,                    % sets the caption-position to bottom
  commentstyle=\color{mygreen},    % comment style
  deletekeywords={...},            % if you want to delete keywords from the given language
  escapeinside={\%*}{*)},          % if you want to add LaTeX within your code
  extendedchars=true,              % lets you use non-ASCII characters; for 8-bits encodings only, does not work with UTF-8
  frame=none,	                   % adds a frame around the code
  keepspaces=true,                 % keeps spaces in text, useful for keeping indentation of code (possibly needs columns=flexible)
  keywordstyle=\color{blue},       % keyword style
  language=VHDL,                   % the language of the code
  morekeywords={*,...},            % if you want to add more keywords to the set
  numbers=left,                    % where to put the line-numbers; possible values are (none, left, right)
  numbersep=5pt,                   % how far the line-numbers are from the code
  numberstyle=\tiny\color{mygray}, % the style that is used for the line-numbers
  rulecolor=\color{black},         % if not set, the frame-color may be changed on line-breaks within not-black text (e.g. comments (green here))
  showspaces=false,                % show spaces everywhere adding particular underscores; it overrides 'showstringspaces'
  showstringspaces=false,          % underline spaces within strings only
  showtabs=false,                  % show tabs within strings adding particular underscores
  stepnumber=1,                    % the step between two line-numbers. If it's 1, each line will be numbered
  stringstyle=\color{mymauve},     % string literal style
  tabsize=2,                       % sets default tabsize to 2 spaces
  title=\lstname                   % show the filename of files included with \lstinputlisting; also try caption instead of title
}


